%--------------------------------------------------------------------------------------------------------------
% header.tex
%--------------------------------------------------------------------------------------------------------------

%--------------------------------------------------------------------------------------------------------------
% Art des Dokumentes, Layout, Papierformat, Schriftgröße
%--------------------------------------------------------------------------------------------------------------
\documentclass[12pt,					              % Grundschriftgöße
							 oneside,				              % einseitiges Dokument
							 a4paper,				              % Papiergröße
							 toc=listof,		              % Verzeichnis (Abbildungen etc.) in das Inhaltsverzeichnis
							 toc=bibliography,			      % Literaturverzeichnis ins Inhaltsverzeichnis
							 fleqn,					              % Mathematische Formeln linksbündig darstellen
							 numbers=noenddot,
							 headsepline]	          % Punkt am Ende der Nummerierung des Inhaltsverzeichnisses entfernen
							 {scrreprt}

%--------------------------------------------------------------------------------------------------------------
% Daten für die Titelseite
%--------------------------------------------------------------------------------------------------------------
\title{\Kategorie}
\author{\Verfasser}
\date{\today{}}
							 
%--------------------------------------------------------------------------------------------------------------
% Schriftarten anpassen
%--------------------------------------------------------------------------------------------------------------
\setkomafont{sectioning}{\rmfamily\bfseries}					% Titelzeilen
\setkomafont{caption}{\small}													% Schrift für Caption
\setkomafont{captionlabel}{\sffamily\bfseries\small}	% Schrift für 'Abbildung'
\setkomafont{chapterentry}{\small\bfseries}						% Schrift für Inhaltsverzeichnis
\setkomafont{chapter}{\large\bfseries}								% Schrift für Kapitel
\setkomafont{section}{\normalsize}										% Schrift für Section
\setkomafont{subsection}{\normalsize}									% Schrift für Subsection
							 
%--------------------------------------------------------------------------------------------------------------
% Konstanten festlegen 
%--------------------------------------------------------------------------------------------------------------
\newcommand{\Kategorie}{Bachelor-Thesis}
\newcommand{\Titel}{Untersuchung des ETL-Tools Tensei-Data und dessen Einsatz im Prozess der Datenmigration bei der SIV.\@AG}
\newcommand{\Verfasser}{Martin Pohl}
\newcommand{\Geburtstag}{13. April 1981}
\newcommand{\Geburtsort}{Rostock}
\newcommand{\Betreuer}{Prof. Dr.-Ing. Antje Raab-D\"usterh\"oft}
\newcommand{\ZweitBetreuer}{Betriebswirt (VWA) Detlef Herold}

%--------------------------------------------------------------------------------------------------------------
% benötigte Pakete
%--------------------------------------------------------------------------------------------------------------
\usepackage[utf8]{inputenc}             % Zeichensatzkodierung. Direkte Angabe von Umlauten im Dokument. 
\usepackage[english, ngerman]{babel}    % Deutsche und Englische Sprachanpassung
\usepackage[T1]{fontenc}                % Silbentrennung bei Sonderzeichen
\usepackage{lmodern}					          % bietet neuere Schriften, sieht besser aus im Acrobat Reader
\usepackage{graphicx}                   % einfaches Einbinden von Grafiken
\usepackage{subfigure}				          % erweiterte Darstellung von Bildern
\usepackage{listings}                   % ermöglicht Einbinden von Quelltexten
\usepackage{amsmath,amssymb}	          % erweiterter Formelsatz und zusätzliche Mathe-Symbole
\usepackage{booktabs}					          % professionelle Tabellen setzen, typographisch richtig
\usepackage{shortvrb}					          % benutzen der Verbatim-Umgebung, stellt Quelltexte exakt da
\usepackage{setspace}					          % Package zum Kontrollieren von Leerräumen
\usepackage{color,moreverb}		          % Farben
\usepackage{scrhack}                    % behebt Warnungen die mit float und dem Koma-Skript zu tun haben
\usepackage{blindtext}                  % zum Einbinden von Blindtexten
\usepackage[plainheadsepline, nouppercase,]{scrpage2}                   % Koma-Skript konforme Möglichkeit Kopf- und Fußzeile anzupassen
\usepackage{graphicx}
\usepackage{url}


%--------------------------------------------------------------------------------------------------------------
% Definition Seitenränder
%--------------------------------------------------------------------------------------------------------------
\usepackage[left=3cm,					  % linker Rand
						right=3cm,				  % rechter Rand
						top=1.5cm,				  % oberer Rand
						bottom=1.5cm,			  % unterer Rand
						includeheadfoot,	  % bezieht die Kopf- und Fußzeile mit ein
						bindingoffset=0cm,]	% Bundsteg
						{geometry}

%--------------------------------------------------------------------------------------------------------------
% Metadaten für die PDF-Erzeugung
%--------------------------------------------------------------------------------------------------------------
\usepackage[pdftex,
						pdfauthor={\Verfasser},	                                    % Name des Autors
						pdftitle={\Titel},			                                    % Name der Arbeit
						pdfcreator={Latexmk, KOMA-Script},                          % Von was erzeugt
						pdfsubject={\Kategorie},	                                  % Was für eine Arbeit ist es
						pdfkeywords={\Titel},
						plainpages=false,
						hypertexnames=false,
						pdfpagelabels]{hyperref}

%--------------------------------------------------------------------------------------------------------------
% Farbe für Links in PDF-Dokumenten definieren
%--------------------------------------------------------------------------------------------------------------
\definecolor{LinkColor}{rgb}{0,0,0.5}	% Festlegen einer neuen Farbe

\hypersetup{colorlinks=true,			% farbliche Links
						breaklinks=true,			% Zeilenumbruch erlauben
						linkcolor=black,			% Farbe für interne Links
						citecolor=black,			% Farbe für Links zum Literaturverzeichnis
						filecolor=LinkColor,	% Farbe für externe Dateilinks
						menucolor=LinkColor,	%
						urlcolor=LinkColor}		% Farbe für externe Links

%--------------------------------------------------------------------------------------------------------------
% Darstellung des Literaturverzeichnisses einstellen
%--------------------------------------------------------------------------------------------------------------
\bibliographystyle{alphadin}	% Stil des Literaturverzeichnisses (hier nach DIN 1505)

%--------------------------------------------------------------------------------------------------------------
% Darstellung des Glossars und des Abkürzungsverzeichnisses einstellen
%--------------------------------------------------------------------------------------------------------------
\usepackage[nomain,style=listdotted,nonumberlist,acronym,toc]{glossaries}
\usepackage{glossary-longragged}
\renewcommand{\glspostdescription}{}        % Punkt am Ende der vollständigen Beschreibung entfernen
%\loadglsentries{verzeichnisse/abkuerzungen} % Alternativdatei für die Definition der Abkürzungen
\makeglossaries
\newacronym{erp}{ERP}{Enterprise Resource Planning}

\newacronym{etl}{ETL}{Extraction, Transformation, Loading}

\newacronym{dbms}{DBMS}{Datenbank-Management-System}

\newacronym{plsql}{PL/SQL}{Procedural Language/Structured Query Language}

\newacronym{rdbms}{RDBMS}{Relational Database Management System}




\newacronym{lan}{LAN}{Local Area Network}

\newacronym{din}{DIN}{Deutsches Institut für Normung}

\newacronym{iso}{ISO}{International Standards Organization}

\newacronym{ieee}{IEEE}{Institute of Electronic and Electrotechnical Engineers}

\newacronym{svm}{svm}{support vector machine}

\newacronym{vcs}{VCS}{Version Control System}

\newacronym{swe}{SWE}{Softwareentwicklung}

\newacronym{ide}{IDE}{integrierte Entwicklungsumgebung}

\newacronym{gui}{GUI}{Graphical User Interface}

\newacronym{url}{URL}{Uniform Resource Locator}

\newacronym{www}{WWW}{World Wide Web}

\newacronym{http}{HTTP}{Hyper Text Transfer Protocol}

\newacronym{cwi}{CWI}{Centrum voor Wiskunde en Informatica}

\newacronym{gpl}{GPL}{General Public License}

\newacronym{ocr}{OCR}{optische Zeichenerkennung}

\newacronym{xml}{XML}{Extensible Markup Language}

\newacronym{html}{HTML}{Hypertext Markup Language}

\newacronym{xmlp}{XMLP}{XML Path Language}

\newacronym{dom}{DOM}{Document Object Model}

\newacronym{sax}{SAX}{Simple API for XML}

\newacronym{uic}{UIC}{Qt user interface compiler}
				% die Datei abkürzungen.tex zum Verwalten der Abkürzungen verwenden
%--------------------------------------------------------------------------------------------------------------
% Abbildung in Bild umbenennen
%--------------------------------------------------------------------------------------------------------------
\addto{\captionsngerman}{
	\renewcommand*{\figurename}{Bild}}
	
	
\DeclareGraphicsExtensions{.pdf,.png,.jpg}

%--------------------------------------------------------------------------------------------------------------
% Quellcodeformatierung
%--------------------------------------------------------------------------------------------------------------

\lstloadlanguages{Matlab,[Visual]C++,C, Python}

%\definecolor{lbcolor}{white}{0.9}			% Farbe für den Hintergrund definieren				
\definecolor{darkblue}{rgb}{0,0,.6}		% Farbe für Schlüsselwörter
\definecolor{darkred}{rgb}{.6,0,0}		% Farbe für Strings
\definecolor{darkgreen}{rgb}{0,.6,0}	% Farbe für Kommentare

%\lstset{language=Matlab,					% Programmiersprache der Listings
%				numbers=left,							% Zeilennummern links angeben
%				stepnumber=1,							% in welchem Abstand sollen Zeilennummern angeben werden (1 2 3..)
%				numbersep=5pt,
%				numberstyle=\tiny,				% grösse der Nummern
%				breaklines=true,					% Zeilenumbruch zulassen
%				breakautoindent=true,
%				postbreak=\space,
%				tabsize=2,								% Tabulator auf 2 setzen
%				basicstyle=\ttfamily\footnotesize,
%				showspaces=false,					% leerzeichen nicht anzeigen
%				showstringspaces=false,		% keine Leerzeichen bei Strings anzeigen
%				extendedchars=true,
%				backgroundcolor=\color{lbcolor}}	% Hintergrundfarbe des Listings

\lstset{language=Python,								% Programmiersprache der Listings
				%alsolanguage=Matlab,			% alternative Programmiersprache der Listings
				frame=none,								% keinen Rahmen
				frameround=ffff,					% wenn ein Rahmen dargestellt werden soll, sind die Ecken spitz
				captionpos=b,							% Position der Benennung
				numbers=left,							% Zeilennummern links angeben
				stepnumber=1,							% in welchem Abstand sollen Zeilennummern angeben werden (1 2 3..)
				numbersep=5pt,						% Abstand zwischen Nummerierung und Listing
				numberstyle=\tiny,				% grösse der Nummern
				breaklines=true,					% Zeilenumbruch zulassen
				breakautoindent=true,
				postbreak=\space,
				tabsize=4,								% Tabulator auf 4 setzen
				escapechar=\$,
				basicstyle=\scriptsize\ttfamily,
				keywordstyle=\color{darkblue}\bfseries\ttfamily,	% Darstelung der Schlüsselwörter
				stringstyle=\ttfamily\color{darkred},  						% Darstellung der Strings
				commentstyle=\itshape\color{darkgreen},						% Darstellung der Kommentare
				showspaces=false,					% leerzeichen nicht anzeigen
				showstringspaces=false,		% keine Leerzeichen bei Strings anzeigen
				xleftmargin=.52cm,
				xrightmargin=.52cm,				
				backgroundcolor=\color{white}}	% Hintergrundfarbe des Listings}

%--------------------------------------------------------------------------------------------------------------
% Darstellung Kopf- und Fußzeile einstellen
%--------------------------------------------------------------------------------------------------------------

\pagestyle{scrheadings}
\automark[chapter]{chapter}
\clearscrheadfoot
\renewcommand{\headfont}{\normalfont}
\renewcommand*{\chaptermarkformat}{\chapapp\ \thechapter.\ \ }% Nummer in Kopfzeile verschwindet
\rohead[\headmark]{\headmark}
\rofoot[\pagemark]{\pagemark}
